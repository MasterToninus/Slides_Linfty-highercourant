\documentclass[10pt,a4paper]{article}
\usepackage[utf8]{inputenc}
\usepackage{amsmath}
\usepackage{amsfonts}
\usepackage{amssymb}
\usepackage[left=2cm,right=2cm,top=2cm,bottom=2cm]{geometry}

\title{On the $L_\infty$-algebra associated to higher Courant algebroids}
\author{Antonio Michele Miti}

\begin{document}
\maketitle

\begin{abstract}
In this talk, we  discuss the graded geometry aspects underlying the construction of the $L_\infty$ algebra associated with a higher Courant algebroid as introduced by Zambon in \cite{Zambon2012}.
The key point is to notice that one can associate to any smooth manifold $M$ a graded manifold $T^\ast[r]T[1]M$ and how the corresponding algebra of function $\mathcal{C}$ is naturally $n$-Poisson.

This talk is based on \cite{Zambon2012}, Section 8.
Most of our conventions in graded geometry are taken from \cite[\S 2]{Cattaneo2006} \cite[\S 2.4]{Schatz2009}.
The basic notions on $n$-Poisson algebra are taken from \cite{Cattaneo2006a}.
Several results on the algebraic properties on multiderivations in the ungraded settings can be found in \cite{Laurent-Gengoux2013}.
\end{abstract}



\cite{*}
%------------------------------------------------------------------------------------------------
% Bibliography (BibTex)
% https://arxiv.org/hypertex/bibstyles/
%------------------------------------------------------------------------------------------------
			\bibliographystyle{hep}
			\bibliography{biblio}
%------------------------------------------------------------------------------------------------

\end{document}